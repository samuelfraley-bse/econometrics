\documentclass[12pt,a4paper]{article}

% packages.tex  (limpio, conserva tus paquetes originales)

% Custom vars
\def\gr{...}      % Set the group number cons. across doc
\def\nass{1}    % Set the assignment number cons. across doc
\def\cl{Econometrics }   % Define the class

% --------------------------
% Encoding & page geometry
% --------------------------
\usepackage[utf8]{inputenc}
\usepackage[T1]{fontenc}
\usepackage[
  a4paper,
  total={170mm,257mm},
  left=25mm,
  right=25mm,
  top=25mm,
  bottom=25mm
]{geometry}

% --------------------------
% Graphics, images, svg
% --------------------------
\usepackage{graphicx}
\usepackage{svg}                 
\graphicspath{{Latex/imgs/}}

% --------------------------
% Math, plots, tikz
% --------------------------
\usepackage{amsmath,amssymb}
\usepackage{mathtools}
\usepackage{pgfplots}
\pgfplotsset{width=12cm,compat=newest}
\usepackage{tikz}
\usetikzlibrary{shapes.geometric, arrows, positioning}

% --------------------------
% Figures, captions, tables
% --------------------------
\usepackage[font=small,labelfont=bf]{caption}
\usepackage{subcaption}          
\usepackage{booktabs}
\usepackage{makecell}
\usepackage{multirow}
\usepackage{threeparttable}
\usepackage{tabularx}
\usepackage{float}
\usepackage{wrapfig}
\usepackage{transparent}

% --------------------------
% Code highlighting (conservo ambos: minted y listings)
% --------------------------
% minted requiere --shell-escape al compilar (ver nota abajo).
\usepackage[newfloat]{minted}     % Code highlighting (más potente)
\newenvironment{code}{\captionsetup{type=listing}}{}
\SetupFloatingEnvironment{listing}{name=Code}

\usepackage{listings}            
\lstset{
  language=Python,
  basicstyle=\ttfamily\footnotesize,
  backgroundcolor=\color{gray!10},
  frame=single,
  keywordstyle=\color{blue},
  commentstyle=\color{green!50!black},
  stringstyle=\color{red!70!black}
}

\setminted{
    fontsize=\small,
    breaklines=true,
    linenos=true,
    autogobble=true,
    mathescape=true,
    breakanywhere=true,
    samepage=false
}

% --------------------------
% Utilities
% --------------------------
\usepackage{xcolor}
\usepackage{url}
\usepackage{enumitem}
\usepackage{xspace}
\usepackage{multicol}

% --------------------------
% Header / footer
% --------------------------
\usepackage{fancyhdr}
\fancyhf{}                           
\fancyhead[L]{DSDM - BSE}
\fancyhead[C]{\cl}
\fancyhead[R]{Assignment \nass}
\renewcommand{\headrulewidth}{0.4pt}
\fancyfoot[C]{\thepage}
\pagestyle{fancy}

% --------------------------
% Extras / includes / pdfpages
% --------------------------
\usepackage{pdfpages}
\usepackage{newclude}               % si usas \includeonly/\newinclude
\usepackage{csquotes}

% --------------------------
% Hyperref (una sola carga) + metadata
% --------------------------
\usepackage{hyperref}
\hypersetup{
    pdftitle    = {\cl - Assignment~\nass},
    pdfsubject  = {This is a submission in the DSDM Masters at BSE.},
    pdfauthor   = {Group~\gr},
    pdfcreator  = {Overleaf},
    pdfstartview= FitH
}

% --------------------------
% Bibliografía (biblatex)
% --------------------------
\usepackage[backend=biber, style=authoryear, hyperref=true]{biblatex}
\usepackage{csquotes}               
\DeclareCiteCommand{\cite}
  {\usebibmacro{prenote}}
  {\usebibmacro{citeindex}%
   \printnames{labelname}%
   \space(\printfield{year})}
  {\multicitedelim}
  {\usebibmacro{postnote}}
\DeclareNameAlias{labelname}{family-given}
\renewcommand*{\nameyeardelim}{\addcomma\space}
\AtEveryCitekey{\ifciteseen{}{\defcounter{maxnames}{1}}}
\addbibresource{Latex/chapters/references.bib}

% --------------------------
% Espaciado / estilo de párrafo
% --------------------------
\setlength{\parskip}{0.3\baselineskip}
\setlength{\parindent}{0pt}
\linespread{1.15}

% --------------------------
% List of code (adecuado para article -> uso section)
% --------------------------
\renewcommand{\listoflistings}{
  \cleardoublepage
  \addcontentsline{toc}{section}{List of Code} 
  \listof{listing}{List of Code}
}


\title{Assignment 1}
\author{
  Samuel Fraley \\
  Dmitrii Kuptsov \\
  Felipe Manzi
}
\date{\today}

\begin{document}

\maketitle
\tableofcontents
\newpage

\section*{Question 1}
\begin{enumerate}[label=(\alph*)]
  \item Describe, in one precise sentence, what term (1) captures.

  Term 1, $E(y_{i} \mid d_i = 1)$, captures the expected value of of $y_{i}$ given $d_{i} = 1$, specifically the expected monthly pay of a person given they have a college degree ($d_{i} = 1$)

  \item What sign would you expect for (1) – (2)? Briefly justify.

  Term 1 is the expected monthly pay of individuals with a college degree, while Term 2 is the expected monthly pay of those without a college degree. Generally we would expect those with a college degree to, on average, have a higher monthly pay, so Term 1 would be larger than Term 2. As a result, we would expect (1) - (2) to be positive.
  
  \item Describe what does term (3) capture. Be specific.
  
  \item Explain how the sample can be used to provide information for term (1) – (2) but not (3) or (4).
  
  \item What would it mean for term (4) to be positive?
  
  \item What are the implications of term (4) being positive for measuring the effect of a degree on earnings?
\end{enumerate}

\section*{Question 2}
Discrete random variables $X$ and $Y$ can take 10 equally likely pairs:
\[
(1, 2), (1, 4), (1, 6), (2, 1), (2, 3), (2, 5), (3, 2), (3, 4), (3, 6), (3, 8).
\]
Verify the Law of Total Expectations: $E[E(Y|X)] = E(Y)$.  
\textbf{Answer:} \ldots

\section*{Question 3}
Variable $X \sim N(0,1)$ and $Y = X^2 - 1$. Show how this example illustrates that uncorrelated $\not\Rightarrow$ independent.  
\textbf{Answer:} \ldots

\section*{Question 4}
Joint pdf: $f(x, y) = \frac{3(x^2 + y)}{11}$ for $0 \leq x \leq 2$, $0 \leq y \leq 1$.

\begin{enumerate}[label=(\alph*)]
  \item Find the best linear approximation to the CEF.  
  \item Plot the CEF and linear fit (include code + figure).
\end{enumerate}

\section*{Question 5}
Simple regression model: $y = \beta_1 + \beta_2 x^2 + \epsilon$.

\begin{enumerate}[label=(\alph*)]
  \item Prove $\beta_1,\beta_2$ solve the least squares problem.  
  \item Reverse regression $x^2 = \alpha_1 + \alpha_2 y + v$: write expressions for $\alpha_1,\alpha_2$.  
  \item When does $\alpha_2 = 1/\beta_2$? Justify.
\end{enumerate}

\end{document}
