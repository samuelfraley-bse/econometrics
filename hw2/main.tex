\documentclass[12pt,a4paper]{article}

% packages.tex  (limpio, conserva tus paquetes originales)

% Custom vars
\def\gr{...}      % Set the group number cons. across doc
\def\nass{1}    % Set the assignment number cons. across doc
\def\cl{Econometrics }   % Define the class

% --------------------------
% Encoding & page geometry
% --------------------------
\usepackage[utf8]{inputenc}
\usepackage[T1]{fontenc}
\usepackage[
  a4paper,
  total={170mm,257mm},
  left=25mm,
  right=25mm,
  top=25mm,
  bottom=25mm
]{geometry}

% --------------------------
% Graphics, images, svg
% --------------------------
\usepackage{graphicx}
\usepackage{svg}                 
\graphicspath{{Latex/imgs/}}

% --------------------------
% Math, plots, tikz
% --------------------------
\usepackage{amsmath,amssymb}
\usepackage{mathtools}
\usepackage{pgfplots}
\pgfplotsset{width=12cm,compat=newest}
\usepackage{tikz}
\usetikzlibrary{shapes.geometric, arrows, positioning}

% --------------------------
% Figures, captions, tables
% --------------------------
\usepackage[font=small,labelfont=bf]{caption}
\usepackage{subcaption}          
\usepackage{booktabs}
\usepackage{makecell}
\usepackage{multirow}
\usepackage{threeparttable}
\usepackage{tabularx}
\usepackage{float}
\usepackage{wrapfig}
\usepackage{transparent}

% --------------------------
% Code highlighting (conservo ambos: minted y listings)
% --------------------------
% minted requiere --shell-escape al compilar (ver nota abajo).
\usepackage[newfloat]{minted}     % Code highlighting (más potente)
\newenvironment{code}{\captionsetup{type=listing}}{}
\SetupFloatingEnvironment{listing}{name=Code}

\usepackage{listings}            
\lstset{
  language=Python,
  basicstyle=\ttfamily\footnotesize,
  backgroundcolor=\color{gray!10},
  frame=single,
  keywordstyle=\color{blue},
  commentstyle=\color{green!50!black},
  stringstyle=\color{red!70!black}
}

\setminted{
    fontsize=\small,
    breaklines=true,
    linenos=true,
    autogobble=true,
    mathescape=true,
    breakanywhere=true,
    samepage=false
}

% --------------------------
% Utilities
% --------------------------
\usepackage{xcolor}
\usepackage{url}
\usepackage{enumitem}
\usepackage{xspace}
\usepackage{multicol}

% --------------------------
% Header / footer
% --------------------------
\usepackage{fancyhdr}
\fancyhf{}                           
\fancyhead[L]{DSDM - BSE}
\fancyhead[C]{\cl}
\fancyhead[R]{Assignment \nass}
\renewcommand{\headrulewidth}{0.4pt}
\fancyfoot[C]{\thepage}
\pagestyle{fancy}

% --------------------------
% Extras / includes / pdfpages
% --------------------------
\usepackage{pdfpages}
\usepackage{newclude}               % si usas \includeonly/\newinclude
\usepackage{csquotes}

% --------------------------
% Hyperref (una sola carga) + metadata
% --------------------------
\usepackage{hyperref}
\hypersetup{
    pdftitle    = {\cl - Assignment~\nass},
    pdfsubject  = {This is a submission in the DSDM Masters at BSE.},
    pdfauthor   = {Group~\gr},
    pdfcreator  = {Overleaf},
    pdfstartview= FitH
}

% --------------------------
% Bibliografía (biblatex)
% --------------------------
\usepackage[backend=biber, style=authoryear, hyperref=true]{biblatex}
\usepackage{csquotes}               
\DeclareCiteCommand{\cite}
  {\usebibmacro{prenote}}
  {\usebibmacro{citeindex}%
   \printnames{labelname}%
   \space(\printfield{year})}
  {\multicitedelim}
  {\usebibmacro{postnote}}
\DeclareNameAlias{labelname}{family-given}
\renewcommand*{\nameyeardelim}{\addcomma\space}
\AtEveryCitekey{\ifciteseen{}{\defcounter{maxnames}{1}}}
\addbibresource{Latex/chapters/references.bib}

% --------------------------
% Espaciado / estilo de párrafo
% --------------------------
\setlength{\parskip}{0.3\baselineskip}
\setlength{\parindent}{0pt}
\linespread{1.15}

% --------------------------
% List of code (adecuado para article -> uso section)
% --------------------------
\renewcommand{\listoflistings}{
  \cleardoublepage
  \addcontentsline{toc}{section}{List of Code} 
  \listof{listing}{List of Code}
}


\title{Assignment 2}
\author{
  Samuel Fraley \\
  Eric Gutierrez \\
  Corneel Moons
}
\date{October 9, 2025}

\begin{document}

\maketitle
\tableofcontents
\newpage

\section*{Question 1: Distances and Scale Transformations}
\begin{enumerate}[label=(\alph*)]
  \item Compute the squared Euclidean distance $d_E(v_0, \bar{v})$ and the squared Mahalanobis distance $d_M(v_0, \bar{v})$ between
  \[
  v_0 = \begin{bmatrix}2 \\ 5 \\ 3\end{bmatrix},
  \qquad
  \bar{v} = \begin{bmatrix}0 \\ 1 \\ 1\end{bmatrix},
  \qquad
  S = 
  \begin{bmatrix}
  5 & 1 & 2 \\
  1 & 6 & 1 \\
  2 & 1 & 7
  \end{bmatrix}.
  \]
  Include the functions (in \texttt{R}/\texttt{Python}) you used to compute both distances and report their values here.

  \item Suppose $v_1$ is rescaled so that each value is multiplied by $10$.  
  (i) Report the new $\bar{v}$, $v_0$, and $S$ matrices.  
  (ii) Recalculate $d_E(v_0, \bar{v})$ and $d_M(v_0, \bar{v})$ using your functions.  
  Comment briefly on the effect of the rescaling on each distance measure.
\end{enumerate}

\newpage
\section*{Question 2: Effect of Units on OLS Estimation}
We estimate the model:
\[
y = \beta_1 + \beta_2 \ln(x_2) + \beta_3 x_3 + \varepsilon,
\quad \hat{\beta} = (X'X)^{-1} X'y.
\]

\begin{enumerate}[label=(\alph*)]
  \item Define matrix $A$ such that $X^* = X A$ after rescaling $x_2$ and $x_3$ by constants $a$ and $b$ respectively.
  \item Derive the relationship between $\hat{\beta}^*$ and $\hat{\beta}$.  
  Comment on how a change in the units of measurement affects each estimated parameter.  
  You may use symbolic computation (\texttt{sympy} in Python or equivalent) to show the steps.
\end{enumerate}

\newpage
\section*{Question 3: Sample Verification of the Linear Model Equivalence}
Consider the sample regression
\[
\ln(wage) = \beta_1 + \beta_2 \, educ + \varepsilon,
\]
using data \texttt{dataFig311.csv} (Angrist \& Pischke, 1980s U.S. sample).

\begin{enumerate}[label=(\alph*)]
  \item Estimate the regression using OLS and report $\hat{\beta}_1$ and $\hat{\beta}_2$.
  \item Compute conditional means of $\ln(wage)$ for each level of \texttt{educ},  
  and show that running the regression on these conditional means yields the same fitted line as using all individual observations.  
  Include the weighted averages and plots as appropriate.
\end{enumerate}

\newpage

\newpage
\section*{Question 4: Social Media and Corruption (Enikolopov et al., 2018)}
We analyze the regression (R1):
\[
Corruption = \beta_1 + \beta_2 \ln(gdp) + \beta_3\, smedia + \varepsilon,
\]
using the dataset \texttt{corruption.csv} with $n=35$ countries.

\begin{enumerate}[label=(\alph*)]

\item \textbf{OLS estimation by formula.}  
Estimate parameters with the closed-form expression
\[
\hat{\beta} = (X'X)^{-1}X'y,
\]
and compute $SSE(\hat{\beta})$, $y'M_Xy$, and $R^2$.

(i) Calculate the associated sum of squared errors, $SSE(\hat{\beta})$, and $R^2$. 
Provide the values of $\hat{\beta}$, $SSE(\hat{\beta})$, $y'M_Xy$, and $R^2$ as your answer.

\begin{table}[h!]\centering
\begin{tabular}{l r}
\hline
$\hat{\beta}_1$ (Intercept) & 100.57 \\
$\hat{\beta}_2$ ($\ln(gdp)$) & $-0.38$ \\
$\hat{\beta}_3$ ($smedia$) & $-0.00$ \\
\hline
$SSE(\hat{\beta})$ & 154.92 \\
$y'M_Xy$ & 154.92 \\
$R^2$ & 0.75 \\
\hline
\end{tabular}
\caption{Closed–form OLS results for regression (R1).}
\end{table}

\vspace{0.2cm}
(ii) Are you surprised about the value of $SSE(\hat{\beta})$ compared to $y'M_Xy$?

No, we are not surprised that $SSE(\hat{\beta})$  is equal to $y'M_Xy$

% ----------------------------------------------------------------------

\item \textbf{Relative importance via Shapley value decomposition.}

Insert the printed Shapley values and include your generated plot below.  
\begin{center}
\includegraphics[width=0.8\textwidth]{Q4b_shapley_plot.png}
\end{center}
Comment briefly on which regressor contributes more to $R^2$ and interpret economically.

% ----------------------------------------------------------------------

\item \textbf{Frisch–Waugh–Lovell (FWL) regression (R2).}

Write explicitly your (R2) regression expression and verify that $\hat{\beta}_3$ here  
matches the $\hat{\beta}_3$ from (R1).

% ----------------------------------------------------------------------

\item \textbf{Interpretation of the added-variable plot.}  
Explain what variables are on each axis of Figure 1 (from the paper),  
why both axes can take positive and negative values,  
and what the plotted relationship represents in terms of residuals.

% ----------------------------------------------------------------------

\item \textbf{Interpretation of the slope.}  
State—in one rigorous sentence—what the slope of the added-variable plot says  
about the relationship between social media penetration and corruption.
\end{enumerate}











\section*{References}
\begin{itemize}
  \item Angrist, J.D. \& Pischke, J.S. (2009). *Mostly Harmless Econometrics: An Empiricist’s Companion*.  
  \item Enikolopov, R., Makarin, A., Petrova, M. (2018). “Social Media and Corruption.” *American Economic Journal: Applied Economics*.
\end{itemize}

\end{document}
