\documentclass[12pt,a4paper]{article}

% packages.tex  (limpio, conserva tus paquetes originales)

% Custom vars
\def\gr{...}      % Set the group number cons. across doc
\def\nass{1}    % Set the assignment number cons. across doc
\def\cl{Econometrics }   % Define the class

% --------------------------
% Encoding & page geometry
% --------------------------
\usepackage[utf8]{inputenc}
\usepackage[T1]{fontenc}
\usepackage[
  a4paper,
  total={170mm,257mm},
  left=25mm,
  right=25mm,
  top=25mm,
  bottom=25mm
]{geometry}

% --------------------------
% Graphics, images, svg
% --------------------------
\usepackage{graphicx}
\usepackage{svg}                 
\graphicspath{{Latex/imgs/}}

% --------------------------
% Math, plots, tikz
% --------------------------
\usepackage{amsmath,amssymb}
\usepackage{mathtools}
\usepackage{pgfplots}
\pgfplotsset{width=12cm,compat=newest}
\usepackage{tikz}
\usetikzlibrary{shapes.geometric, arrows, positioning}

% --------------------------
% Figures, captions, tables
% --------------------------
\usepackage[font=small,labelfont=bf]{caption}
\usepackage{subcaption}          
\usepackage{booktabs}
\usepackage{makecell}
\usepackage{multirow}
\usepackage{threeparttable}
\usepackage{tabularx}
\usepackage{float}
\usepackage{wrapfig}
\usepackage{transparent}

% --------------------------
% Code highlighting (conservo ambos: minted y listings)
% --------------------------
% minted requiere --shell-escape al compilar (ver nota abajo).
\usepackage[newfloat]{minted}     % Code highlighting (más potente)
\newenvironment{code}{\captionsetup{type=listing}}{}
\SetupFloatingEnvironment{listing}{name=Code}

\usepackage{listings}            
\lstset{
  language=Python,
  basicstyle=\ttfamily\footnotesize,
  backgroundcolor=\color{gray!10},
  frame=single,
  keywordstyle=\color{blue},
  commentstyle=\color{green!50!black},
  stringstyle=\color{red!70!black}
}

\setminted{
    fontsize=\small,
    breaklines=true,
    linenos=true,
    autogobble=true,
    mathescape=true,
    breakanywhere=true,
    samepage=false
}

% --------------------------
% Utilities
% --------------------------
\usepackage{xcolor}
\usepackage{url}
\usepackage{enumitem}
\usepackage{xspace}
\usepackage{multicol}

% --------------------------
% Header / footer
% --------------------------
\usepackage{fancyhdr}
\fancyhf{}                           
\fancyhead[L]{DSDM - BSE}
\fancyhead[C]{\cl}
\fancyhead[R]{Assignment \nass}
\renewcommand{\headrulewidth}{0.4pt}
\fancyfoot[C]{\thepage}
\pagestyle{fancy}

% --------------------------
% Extras / includes / pdfpages
% --------------------------
\usepackage{pdfpages}
\usepackage{newclude}               % si usas \includeonly/\newinclude
\usepackage{csquotes}

% --------------------------
% Hyperref (una sola carga) + metadata
% --------------------------
\usepackage{hyperref}
\hypersetup{
    pdftitle    = {\cl - Assignment~\nass},
    pdfsubject  = {This is a submission in the DSDM Masters at BSE.},
    pdfauthor   = {Group~\gr},
    pdfcreator  = {Overleaf},
    pdfstartview= FitH
}

% --------------------------
% Bibliografía (biblatex)
% --------------------------
\usepackage[backend=biber, style=authoryear, hyperref=true]{biblatex}
\usepackage{csquotes}               
\DeclareCiteCommand{\cite}
  {\usebibmacro{prenote}}
  {\usebibmacro{citeindex}%
   \printnames{labelname}%
   \space(\printfield{year})}
  {\multicitedelim}
  {\usebibmacro{postnote}}
\DeclareNameAlias{labelname}{family-given}
\renewcommand*{\nameyeardelim}{\addcomma\space}
\AtEveryCitekey{\ifciteseen{}{\defcounter{maxnames}{1}}}
\addbibresource{Latex/chapters/references.bib}

% --------------------------
% Espaciado / estilo de párrafo
% --------------------------
\setlength{\parskip}{0.3\baselineskip}
\setlength{\parindent}{0pt}
\linespread{1.15}

% --------------------------
% List of code (adecuado para article -> uso section)
% --------------------------
\renewcommand{\listoflistings}{
  \cleardoublepage
  \addcontentsline{toc}{section}{List of Code} 
  \listof{listing}{List of Code}
}


\title{Assignment 3}
\author{
  Samuel Fraley \\ 
  Daniel Campos \\ 
  Elvis Casco
}
\date{October 18, 2025}

\begin{document}

\maketitle
\tableofcontents
\newpage

% ============================================================
\section*{Question 1: Assumptions for Random Samples}

\begin{enumerate}[label=(\alph*)]
  \item Rewrite assumptions [A1]–[A4] from class for i.i.d. random samples $\{(x_i, y_i)\}$.

  \textit{Hint:} Express each assumption mathematically and briefly justify its meaning under i.i.d. sampling.

  \vspace{1em}
  \textbf{Example structure:}
  \begin{align*}
  \text{[A1]: } & E[\varepsilon_i \mid X_i] = 0 \\
  \text{[A2]: } & Var(\varepsilon_i \mid X_i) = \sigma^2 < \infty \\
  & \vdots
  \end{align*}

  \textit{Insert explanations and intuition below.}
\end{enumerate}

\newpage
% ============================================================
\section*{Question 2: Conditional vs. Unconditional Properties of OLS}

\begin{enumerate}[label=(\alph*)]
  \item 
  (i) Prove that OLS is unconditionally unbiased given $E(\hat{\beta}_k \mid X) = \beta_k$.  
  (ii) Explain in one sentence the interpretation of unconditional unbiasedness.

  \textit{Hint:} Use the law of iterated expectations.  
  \textbf{Insert derivation:}
  \[
  E(\hat{\beta}_k) = E[E(\hat{\beta}_k \mid X)] = \dots
  \]

  \vspace{1em}
  \item 
  Starting from $\text{Var}(\hat{\beta}_k \mid X) \le \text{Var}(\tilde{\beta}_k \mid X)$,  
  show that $\text{Var}(\hat{\beta}_k) \le \text{Var}(\tilde{\beta}_k)$.

  \textit{Hint:} Use the law of total variance:  
  $\text{Var}(Z) = \text{Var}(E[Z \mid W]) + E[\text{Var}(Z \mid W)]$.

  \textbf{Insert derivation here.}
\end{enumerate}

\newpage
% ============================================================
\section*{Question 3: One-Sided Hypothesis Test}

\begin{enumerate}[label=(\alph*)]
  \item Draw the acceptance and rejection regions for $H_0: \beta_2 = 0$ vs. $H_1: \beta_2 < 0$ at $\alpha=5\%$.

  \textit{Include a clear plot below (hand-drawn or generated in R/Python):}
  \begin{center}
  \includegraphics[width=0.7\textwidth]{Q3a_test_region.png}
  \end{center}

  \item Explain the intuition behind the location of the critical region.

  \textit{Write short reasoning below.}
\end{enumerate}

\newpage
% ============================================================
\section*{Question 4: Monte Carlo Simulation (Assig3Q4.R / Assig3Q4.py)}

\begin{enumerate}[label=(\alph*)]
  \item Provide the expression of the DGP (data generating process) used to simulate samples.

  \textit{Hint:} Derive from line 8 in the code, e.g.}
  \[
  y_i = 10 + 5x_{2i} + \varepsilon_i, \quad \varepsilon_i \sim N(0,6^2)
  \]

  \item (i) How many samples does the code generate?  
  (ii) What do all samples have in common?  
  (iii) Describe lines 1–16 (R) or 6–30 (Python) of the code.

  \textit{Insert concise explanation below.}

  \item Describe the function of lines 18–23 (R) or 32–39 (Python).  
  \item Describe the function of lines 25–40 (R) or 41–56 (Python).  
  \item Run the code and report the value of \texttt{ratiog} and include the plot below.
  \begin{center}
  \includegraphics[width=0.8\textwidth]{Q4e_plot.png}
  \end{center}

  \textit{Comment on whether you are surprised by the value of \texttt{ratiog}.}

  \item Explain what the plot illustrates and provide a rigorous interpretation.

  \item Discuss how increasing $M$ from 100 to 10,000 would affect \texttt{ratiog}.  
  \item Explain how changing confidence level (from 0.025/0.975 to 0.005/0.995) affects \texttt{Ratio} and the plot.

  \item What is the purpose of this simulation exercise?  
  \textit{Summarize the econometric concept it illustrates.}
\end{enumerate}

\newpage
% ============================================================
\section*{Question 5: Fair’s Model of U.S. Presidential Elections}

\begin{enumerate}[label=(\alph*)]
  \item For each of Fair’s four conditions, identify the relevant parameter(s) and expected sign.

  \textbf{Example Table:}
  \begin{table}[h!]\centering
  \begin{tabular}{lcc}
  \hline
  Condition & Parameter & Expected Sign \\
  \hline
  Incumbent running again & $\beta_3$ (DPER) & $+$ \\
  Duration in power & $\beta_4$ (DUR) & $-$ \\
  Republican bias & $\beta_2$ (I) & $-$ \\
  State of economy & $\beta_6$, $\beta_7$, $\beta_8$ & $+$ \\
  \hline
  \end{tabular}
  \caption{Expected signs of parameters in Fair’s model.}
  \end{table}

  \item Estimate the model using Stata with data \texttt{USelections.csv}. Include Stata output below.  
  \textit{Example:}
  \begin{center}
  \includegraphics[width=0.8\textwidth]{Q5b_stata_output.png}
  \end{center}

  \item Define and compute:
  \begin{enumerate}[label=(\roman*)]
    \item $se(\hat{\beta}_6)$  
    \item $t$–value and $p$–value for regressor $I$
  \end{enumerate}
  \textit{Include Python/R calculations and verify against Stata.}

  \item Test $H_0:\beta_7=0$ vs. $H_1:\beta_7\neq 0$ at 5\% significance.  
  Draw acceptance and rejection regions.
  \begin{center}
  \includegraphics[width=0.7\textwidth]{Q5d_ttest_plot.png}
  \end{center}

  \item Draw the p–value region and indicate the minimum significance level that makes $\beta_7$ significant.  
  \item Test joint significance of $(G\cdot I)$, $(P\cdot I)$, and $(Z\cdot I)$ (``It’s the economy, stupid!’’).  
  Include Stata output and steps.

  \item Compute prediction error for 2024 using given $V\!P=49.25$, $G=1.7$, $P=4.54$, $Z=4$.  
  Interpret and comment on Fair’s predictive method.
\end{enumerate}

\newpage
% ============================================================
\section*{References}
\begin{itemize}
  \item Angrist, J.D. \& Pischke, J.S. (2009). \textit{Mostly Harmless Econometrics: An Empiricist’s Companion}.
  \item Fair, R. (2024). \textit{U.S. Presidential Vote Equation}. Yale University, \url{https://fairmodel.econ.yale.edu/}.
  \item Enikolopov, R., Makarin, A., Petrova, M. (2018). “Social Media and Corruption.” \textit{American Economic Journal: Applied Economics}.
\end{itemize}

\end{document}
