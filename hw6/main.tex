\documentclass[15pt,a4paper]{article}
\usepackage{verbatim} 
\usepackage{graphicx} 

% packages.tex  (limpio, conserva tus paquetes originales)

% Custom vars
\def\gr{...}      % Set the group number cons. across doc
\def\nass{1}    % Set the assignment number cons. across doc
\def\cl{Econometrics }   % Define the class

% --------------------------
% Encoding & page geometry
% --------------------------
\usepackage[utf8]{inputenc}
\usepackage[T1]{fontenc}
\usepackage[
  a4paper,
  total={170mm,257mm},
  left=25mm,
  right=25mm,
  top=25mm,
  bottom=25mm
]{geometry}

% --------------------------
% Graphics, images, svg
% --------------------------
\usepackage{graphicx}
\usepackage{svg}                 
\graphicspath{{Latex/imgs/}}

% --------------------------
% Math, plots, tikz
% --------------------------
\usepackage{amsmath,amssymb}
\usepackage{mathtools}
\usepackage{pgfplots}
\pgfplotsset{width=12cm,compat=newest}
\usepackage{tikz}
\usetikzlibrary{shapes.geometric, arrows, positioning}

% --------------------------
% Figures, captions, tables
% --------------------------
\usepackage[font=small,labelfont=bf]{caption}
\usepackage{subcaption}          
\usepackage{booktabs}
\usepackage{makecell}
\usepackage{multirow}
\usepackage{threeparttable}
\usepackage{tabularx}
\usepackage{float}
\usepackage{wrapfig}
\usepackage{transparent}

% --------------------------
% Code highlighting (conservo ambos: minted y listings)
% --------------------------
% minted requiere --shell-escape al compilar (ver nota abajo).
\usepackage[newfloat]{minted}     % Code highlighting (más potente)
\newenvironment{code}{\captionsetup{type=listing}}{}
\SetupFloatingEnvironment{listing}{name=Code}

\usepackage{listings}            
\lstset{
  language=Python,
  basicstyle=\ttfamily\footnotesize,
  backgroundcolor=\color{gray!10},
  frame=single,
  keywordstyle=\color{blue},
  commentstyle=\color{green!50!black},
  stringstyle=\color{red!70!black}
}

\setminted{
    fontsize=\small,
    breaklines=true,
    linenos=true,
    autogobble=true,
    mathescape=true,
    breakanywhere=true,
    samepage=false
}

% --------------------------
% Utilities
% --------------------------
\usepackage{xcolor}
\usepackage{url}
\usepackage{enumitem}
\usepackage{xspace}
\usepackage{multicol}

% --------------------------
% Header / footer
% --------------------------
\usepackage{fancyhdr}
\fancyhf{}                           
\fancyhead[L]{DSDM - BSE}
\fancyhead[C]{\cl}
\fancyhead[R]{Assignment \nass}
\renewcommand{\headrulewidth}{0.4pt}
\fancyfoot[C]{\thepage}
\pagestyle{fancy}

% --------------------------
% Extras / includes / pdfpages
% --------------------------
\usepackage{pdfpages}
\usepackage{newclude}               % si usas \includeonly/\newinclude
\usepackage{csquotes}

% --------------------------
% Hyperref (una sola carga) + metadata
% --------------------------
\usepackage{hyperref}
\hypersetup{
    pdftitle    = {\cl - Assignment~\nass},
    pdfsubject  = {This is a submission in the DSDM Masters at BSE.},
    pdfauthor   = {Group~\gr},
    pdfcreator  = {Overleaf},
    pdfstartview= FitH
}

% --------------------------
% Bibliografía (biblatex)
% --------------------------
\usepackage[backend=biber, style=authoryear, hyperref=true]{biblatex}
\usepackage{csquotes}               
\DeclareCiteCommand{\cite}
  {\usebibmacro{prenote}}
  {\usebibmacro{citeindex}%
   \printnames{labelname}%
   \space(\printfield{year})}
  {\multicitedelim}
  {\usebibmacro{postnote}}
\DeclareNameAlias{labelname}{family-given}
\renewcommand*{\nameyeardelim}{\addcomma\space}
\AtEveryCitekey{\ifciteseen{}{\defcounter{maxnames}{1}}}
\addbibresource{Latex/chapters/references.bib}

% --------------------------
% Espaciado / estilo de párrafo
% --------------------------
\setlength{\parskip}{0.3\baselineskip}
\setlength{\parindent}{0pt}
\linespread{1.15}

% --------------------------
% List of code (adecuado para article -> uso section)
% --------------------------
\renewcommand{\listoflistings}{
  \cleardoublepage
  \addcontentsline{toc}{section}{List of Code} 
  \listof{listing}{List of Code}
}


\title{Assignment 6}
\author{
  Daniel Campos \\
  Eric Gutierrez \\
  Samuel Fraley
}
\date{November 17, 2025}

\begin{document}

\maketitle
\tableofcontents
\newpage

\section*{Question 1: Randomized Control Trials}

This question is based on Duflo, Hanna, and Ryan (2012), who evaluate whether teacher
monitoring combined with financial incentives can reduce teacher absenteeism and improve
learning in primary schools.

The NGO Seva Mandir operates non-formal primary schools in rural villages of Rajasthan (India).
Before the program, teacher absenteeism was high (around 35\%). In 2003, Seva Mandir
introduced a teacher incentive program in 57 randomly selected schools. A camera system
was installed to monitor teacher attendance, and teachers were paid according to a nonlinear
function of valid teaching days (at least 5 hours of teaching with at least 8 students).

The program generated an immediate and persistent improvement in attendance in treated schools.

\subsection*{Data}

The dataset \texttt{ps1\_q1.csv} is a simplified version of the original data collected for this RCT.
Each observation corresponds to a visit to one of the study schools (identified by \texttt{schid}).
The variable \texttt{time} equals 1 in the month before the program starts (baseline) and is
greater than 1 in months after the program begins. Schools are randomly assigned to a treatment
group (\texttt{treat}=1) or control group (\texttt{treat}=0). The main outcome variables are
the number of students (\texttt{students}) and teacher attendance (\texttt{teacher\_attendance}).

\subsection*{1.1 Baseline and Experiment Integrity}

Under proper randomization, potential outcomes are independent of treatment status:
\[
Y_{1i}, Y_{0i} \perp\!\!\!\perp D_i,
\]
which implies that the average treatment effect on the treated (ATT) and the average treatment
effect (ATE) coincide: \(\alpha_{ATT} = \alpha_{ATE} = \beta\).

Before analyzing post-treatment outcomes, we should verify that treated and control schools are
similar at baseline.

\begin{enumerate}[label=(\alph*)]
  \item Using only the observations from the month before the program starts (\texttt{time} = 1),
  compute the average teacher attendance and the average number of students per classroom
  separately for the treatment and control groups.

  Replicate the information in Panels A and B of Table 1 in Duflo et al. (2012): produce baseline
  means by treatment status, report appropriate standard errors, and comment on whether the
  randomization appears to have produced comparable groups.

  \item Briefly discuss whether any observed baseline differences are economically and/or statistically
  significant. Explain why this check is important for interpreting the causal effect estimates later.
\end{enumerate}

\subsection*{1.2 Results}

When randomization is valid, the treatment effect can be estimated via the simple regression:
\[
Y_i = \alpha + \beta D_i,
\]
where \(Y_i\) is the outcome of interest (e.g.\ teacher attendance) and \(D_i\) is the treatment
indicator.

\begin{enumerate}[label=(\alph*), resume]
  \item Using post-program data (\texttt{time} > 1), compute and compare average teacher attendance
  in treated and control schools. Replicate the first three columns of Panel A of Table 2 in Duflo
  et al. (2012), reporting the relevant means and differences.

  \item Based on your estimates, discuss whether the incentive program achieved its main goal of
  reducing teacher absenteeism. Comment on the magnitude and statistical significance of the
  estimated treatment effect.
\end{enumerate}

\newpage

\section*{Question 2: Matching}

Jacobson, LaLonde, and Sullivan (1993, JLS) study earnings losses following job displacement.
Using administrative data from Pennsylvania, they document that workers involved in mass
employment reductions suffer long-term earnings losses of roughly 25\% per year. They distinguish
between separations due to mass layoffs and other separations, and use stayers as a control group.

Their model can be written as:
\[
w^A_{it} = \mu_i
+ \sum_{k \ge -4}^{6} \phi_k L_{it}^k
+ \sum_{l \ge -4}^{6} \psi_l M_{it}^l
+ \beta' X_{it}
+ \rho_t + \varepsilon_{it},
\]
where \(w^A_{it}\) is log annual earnings of worker \(i\), \(L_{it}^k\) and \(M_{it}^l\) are sets of
dummies indicating years relative to layoff and mass layoff, \(X_{it}\) is a vector of covariates,
and \(\rho_t\) are time effects.

Couch and Placzek (2010, CP) revisit this question using matching estimators, arguing that
displaced workers are systematically selected, so estimates based only on JLS-type comparisons
may be biased upward.

Let \(D_i=1\) if worker \(i\) is displaced (due to a mass layoff or other separation) and
\(D_i=0\) otherwise, and let \(p(X_i)\) denote the propensity score. The average treatment
effect on the treated (ATT) is:
\[
\alpha_{TT} = \mathbb{E}\Big[
\mathbb{E}[w^A_{1i} \mid D_i = 1, p(X_i)]
- \mathbb{E}[w^A_{0i} \mid D_i = 0, p(X_i)]
\;\Big|\; D_i = 1
\Big].
\]

To compare outcomes relative to a reference year \(t_0\), CP consider a differenced ATT:
\[
\alpha^{D}_{ATT} = \mathbb{E}\Big[
\big(\mathbb{E}[w^A_{1it} \mid D_i = 1, p(X_i)]
      - \mathbb{E}[w^A_{1it_0} \mid D_i = 1, p(X_i)]\big)
-
\big(\mathbb{E}[w^A_{0it} \mid D_i = 0, p(X_i)]
      - \mathbb{E}[w^A_{0it_0} \mid D_i = 0, p(X_i)]\big)
\;\Big|\; D_i = 1
\Big].
\]

Your task is to revisit CP’s findings using a different dataset.

\subsection*{Data}

The dataset \texttt{ps1\_q2.dta} is built from the Veneto Workers Histories (VWH), an administrative
panel including all individuals working in the Italian region of Veneto from 1975–2001. The file
\texttt{ps1\_q2.dta} contains a subsample of workers who, in 1999, either:
\begin{itemize}
  \item experienced a mass employment reduction,
  \item separated from the firm without being part of a mass layoff, or
  \item stayed with the same employer.
\end{itemize}
The panel covers the years 1995–2001. Mass layoffs are defined using the endogenous separation
rate, following JLS and von Wachter, Song, and Manchester (2009). Displaced workers satisfy the
standard requirements in this literature.

\section*{2.1}

By computing the propensity score using gender, 1995 earnings decile, and decade of birth, the sample is divided into 9 blocks, ensuring that within each block the mean propensity score is not different between the control and treatment groups, that is, between non-displaced and displaced workers. However, and as the output indicates, the balancing property is not satisfied, which means that within the blocks created, the covariates are not balanced. Several variables differ significantly between the treatment and control groups, and consequently, the model is not properly accounting for systematic, idiosyncratic, differences between both groups.

\section*{2.2}

The Nearest-Neighbor (NN) method matches individuals in the treatment group with the closest individual in the control group, using the observed values of the latter as a counterfactual to compute the $\alpha TT$ of the former. The Kernel estimator, instead, takes into account all the individuals in the control group, downweighting those that are farther from any given individual in the treatment group. Since both methods operate differently, we expect them to yield different results, as it is confirmed by the results obtained, which are detailed in the following table:

\begin{table}[h!]
\centering
\begin{tabular}{lcccc}
\hline
\textbf{Year} 
& \textbf{NN Estimate} 
& \textbf{NN SE} 
& \textbf{Kernel Estimate} 
& \textbf{Kernel SE} \\
\hline
1996 & -421.06  & 2098.80 & -554.54 & 150.52 \\
1997 & 352.93   & 2226.50 & -648.94 & 232.19 \\
1998 & -293.59  & 2423.26 & -949.65 & 295.29 \\
1999 & -1700.42 & 2680.80 & -1884.44 & 312.05 \\
2000 & -2440.29 & 2713.74 & -2596.17 & 330.81 \\
2001 & 1088.11  & 2714.94 & -1046.38 & 395.99 \\
\hline
\end{tabular}
\end{table}


Overall, the estimates obtained from both methods differ significantly, as demonstrated by the wide gap between the estimated $\alpha TT$ for the year 2001. The estimates from the Kernel method are significantly more precise than those from the NN method, as can be observed by their corresponding standard error measures. Consequently, we consider the estimated obtained through the Kernel method to be more reliable.

\section*{2.3}

Compared to Figure 1, the estimates obtained through NN and Kernel matching follow similar trends, with both figures showing declining earnings after displacement. This confirms that the results from the matching estimators replicate the patterns observed in Figure 1. Additionally, our graph also remarks the reliability of the Kernel method, and highlights the volatility of NN estimates.

\begin{figure}[H]
    \centering
    \includegraphics[width=0.9\textwidth]{figures/att_estimates_plot.png}
    \caption{Matching estimates using NN and Kernel methods.}
    \label{fig:ex3}
\end{figure}

\newpage

\section*{Question 3: Instrumental Variables}

Angrist and Evans (1998) use an IV strategy to analyze how the number of children affects
parents’ labor supply. They find a sizable negative effect for mothers and essentially no effect
for fathers. Here we focus on mothers and on employment (rather than hours worked).

\subsection*{Data}

The dataset \texttt{ps1\_q3.dta} is a subset of the data used by Angrist and Evans (1998) and
contains only mothers. The key variables are:
\begin{itemize}
  \item \texttt{sexk}: sex of the first child,
  \item \texttt{kidcount}: total number of children,
  \item \texttt{agem}: age of the mother,
  \item \texttt{twin\_latest}: indicator equal to 1 if the last birth was a twin birth,
  \item \texttt{blackm}, \texttt{hispm}, \texttt{othracem}: race dummies,
  \item \texttt{workedm}: indicator equal to 1 if the mother is employed.
\end{itemize}

\subsection*{3.1 Baseline Models}

Consider the model:
\[
y_i = \beta_0 + \beta_1 \text{kidcount}_i + X_i' \beta + \varepsilon_i,
\]
where \(y_i\) is the mother’s employment status and \(X_i\) is a vector of controls.

\begin{enumerate}[label=(\alph*)]
  \item Estimate this equation using OLS.

 \begin{table}[H]
  \centering
  \caption{OLS regression}
  \label{tab:probit}
  \small
  \def\sym#1{\ifmmode^{#1}\else\(^{#1}\)\fi}
\begin{tabular}{l*{1}{cccccc}}
\toprule
                    & Coefficient&   Std. err.&      z-stat&     p-value&95 conf. low&95 conf. high\\
\midrule
KIDCOUNT            &  -0.0910744&   0.0009621&      -94.67&       0.000&  -0.0929600&  -0.0891888\\
AGEM                &   0.0146906&   0.0002214&       66.36&       0.000&   0.0142567&   0.0151244\\
blackm              &   0.1506704&   0.0023869&       63.12&       0.000&   0.1459921&   0.1553487\\
hispm               &  -0.0083050&   0.0045233&       -1.84&       0.066&  -0.0171705&   0.0005605\\
othracem            &   0.0275062&   0.0046310&        5.94&       0.000&   0.0184296&   0.0365828\\
Constant            &   0.3376580&   0.0068619&       49.21&       0.000&   0.3242089&   0.3511071\\
\midrule
Number of obs       &    4.00e+05&            &            &            &            &            \\
R-squared           &   0.0343846&            &            &            &            &            \\
Adj R-squared       &   0.0343725&            &            &            &            &            \\
F-statistic         &    2.85e+03&            &            &            &            &            \\
Root MSE            &   0.4870991&            &            &            &            &            \\
\bottomrule
\end{tabular}

\end{table}
  \noindent\textbf{Figure 1:} OLS regression output from Stata

  \item Estimate the same specification using a probit model.

\begin{table}[H]
  \centering
  \caption{Probit regression}
  \label{tab:probit}
  \small
  \def\sym#1{\ifmmode^{#1}\else\(^{#1}\)\fi}
\begin{tabular}{l*{1}{cccccc}}
\toprule
                    & Coefficient&   Std. err.&      z-stat&     p-value&95 conf. low&95 conf. high\\
\midrule
workedm             &            &            &            &            &            &            \\
KIDCOUNT            &  -0.2370727&   0.0025496&      -92.99&       0.000&  -0.2420698&  -0.2320757\\
AGEM                &   0.0382601&   0.0005794&       66.03&       0.000&   0.0371244&   0.0393957\\
blackm              &   0.4039446&   0.0064219&       62.90&       0.000&   0.3913579&   0.4165314\\
hispm               &  -0.0203433&   0.0117697&       -1.73&       0.084&  -0.0434116&   0.0027249\\
othracem            &   0.0717805&   0.0120918&        5.94&       0.000&   0.0480810&   0.0954800\\
Constant            &  -0.4263508&   0.0178661&      -23.86&       0.000&  -0.4613677&  -0.3913339\\
\midrule
Number of obs       &    4.00e+05&            &            &            &            &            \\
Log likelihood      &   -2.67e+05&            &            &            &            &            \\
LR chi2(5)          &    1.40e+04&            &            &            &            &            \\
Prob > chi2         &   0.0000000&            &            &            &            &            \\
Pseudo R2           &   0.0255562&            &            &            &            &            \\
\bottomrule
\end{tabular}


\end{table}

  \noindent\textbf{Figure 2:} Probit regression output from Stata

  \item Discuss whether these approaches (OLS and probit) are appropriate for identifying the causal
  effect of the number of children on mothers’ labor supply. 
  
  Just using OLS or probit alone is probably not the best approach to identifying the casual effect of number of children on labor supply due to endogeneity problems. For example, women with richer partners may not be required to work, and can decide to have more kids, thus showing that family income could influence both number of children and labor force participation. Various other endogeneity problems may arise as we have a relatively simple model (we already give household/family income, but what about access to childcare such as grandparents present, access to contraceptives, etc).
  
\end{enumerate}


\subsection*{3.2 IV Probit}

\begin{enumerate}[label=(\alph*), resume]
  \item Re-estimate the model using an IV probit specification, instrumenting \texttt{kidcount} with
  \texttt{twin\_latest}. 

  Using {twin\_latest} as an instrument for kidcount is a valid choice as having twins is arguably exogenous to the mother's labor supply decision. 
  The occurrence of twins is largely random and not influenced by the mother's employment status or other socio-economic factors. 
  Therefore, it satisfies the relevance condition (it affects the number of children) 
  and the exclusion restriction (it does not directly affect the mother's employment status except through its effect on the number of children).
  
  To test, we run two stage regression. The first stage regresses kidcount on twin\_latest and other controls, and the second stage regresses workedm on the predicted values of kidcount from the first stage and other controls.
  \begin{table}[H]
  \centering
  \caption{First-stage regression}
  \label{tab:probit}
  \small
  \def\sym#1{\ifmmode^{#1}\else\(^{#1}\)\fi}
\begin{tabular}{l*{1}{cccccc}}
\toprule
                    & Coefficient&   Std. err.&      z-stat&     p-value&95 conf. low&95 conf. high\\
\midrule
twin\_latest         &   0.3850044&   0.0099461&       38.71&       0.000&   0.3655104&   0.4044983\\
SEXK                &   0.0138442&   0.0025263&        5.48&       0.000&   0.0088927&   0.0187958\\
AGEM                &   0.0309710&   0.0003598&       86.08&       0.000&   0.0302658&   0.0316761\\
blackm              &   0.3235942&   0.0038811&       83.38&       0.000&   0.3159874&   0.3312011\\
hispm               &   0.4370486&   0.0073863&       59.17&       0.000&   0.4225716&   0.4515256\\
othracem            &   0.1210053&   0.0075927&       15.94&       0.000&   0.1061238&   0.1358868\\
Constant            &   1.5577578&   0.0110493&      140.98&       0.000&   1.5361014&   1.5794142\\
\midrule
Number of obs       &    4.00e+05&            &            &            &            &            \\
R-squared           &   0.0404864&            &            &            &            &            \\
Adj R-squared       &   0.0404720&            &            &            &            &            \\
F-statistic         &    2.81e+03&            &            &            &            &            \\
Root MSE            &   0.7988634&            &            &            &            &            \\
\bottomrule
\end{tabular}

  \end{table}
   
  Here we see that {twin\_latest} satisfies the relevance condition as it is statistically significant in predicting kidcount.
  
  \begin{table}[H]
  \centering
  \caption{IV Probit regression}
  \label{tab:probit}
  \small
  \def\sym#1{\ifmmode^{#1}\else\(^{#1}\)\fi}
\begin{tabular}{l*{1}{cccccc}}
\toprule
                    & Coefficient&   Std. err.&      z-stat&     p-value&95 conf. low&95 conf. high\\
\midrule
workedm             &            &            &            &            &            &            \\
KIDCOUNT            &  -0.0715877&   0.0413285&       -1.73&       0.083&  -0.1525901&   0.0094146\\
SEXK                &   0.0002138&   0.0040526&        0.05&       0.958&  -0.0077291&   0.0081568\\
AGEM                &   0.0328903&   0.0015186&       21.66&       0.000&   0.0299138&   0.0358667\\
blackm              &   0.3473656&   0.0161035&       21.57&       0.000&   0.3158033&   0.3789279\\
hispm               &  -0.0915425&   0.0210846&       -4.34&       0.000&  -0.1328676&  -0.0502174\\
othracem            &   0.0515402&   0.0131292&        3.93&       0.000&   0.0258073&   0.0772730\\
Constant            &  -0.6801591&   0.0638258&      -10.66&       0.000&  -0.8052555&  -0.5550628\\
\midrule
KIDCOUNT            &            &            &            &            &            &            \\
SEXK                &   0.0138442&   0.0025270&        5.48&       0.000&   0.0088914&   0.0187970\\
AGEM                &   0.0309710&   0.0003467&       89.33&       0.000&   0.0302914&   0.0316505\\
blackm              &   0.3235942&   0.0046597&       69.45&       0.000&   0.3144614&   0.3327271\\
hispm               &   0.4370486&   0.0097855&       44.66&       0.000&   0.4178694&   0.4562278\\
othracem            &   0.1210053&   0.0084547&       14.31&       0.000&   0.1044344&   0.1375763\\
twin\_latest         &   0.3850044&   0.0108025&       35.64&       0.000&   0.3638318&   0.4061769\\
Constant            &   1.5577578&   0.0103840&      150.01&       0.000&   1.5374055&   1.5781101\\
\midrule
/                   &            &            &            &            &            &            \\
athrho2\_1           &  -0.1318425&   0.0328383&       -4.01&       0.000&  -0.1962043&  -0.0674807\\
lnsigma2            &  -0.2245740&   0.0020035&     -112.09&       0.000&  -0.2285007&  -0.2206473\\
\midrule
Number of obs       &    4.00e+05&            &            &            &            &            \\
Log likelihood      &   -7.45e+05&            &            &            &            &            \\
Wald chi2           &    5.17e+03&            &            &            &            &            \\
Prob > chi2         &   0.0000000&            &            &            &            &            \\
Pseudo R2           &            &            &            &            &            &            \\
\bottomrule
\end{tabular}

  \end{table}

  In the IV probit, we see that the coefficient on kidcount is -0.0715877, which is larger in magnitude compared to the OLS and probit estimates.
  This suggests that the negative effect of having more children on mothers' employment is more pronounced when accounting for endogeneity using the IV approach.
   

\end{enumerate}

\subsection*{3.3 Marginal Effects}

\begin{enumerate}[label=(\alph*), resume]
  \item Using your preferred IV probit specification, estimate the marginal effect of an additional
  child on the probability that a mother is employed.

  Table~X reports the predicted probability of employment for mothers with 0–5 children, based on the IV probit estimates. 
  The probability falls from about 0.63 for one child to 0.43 for three children, implying that an additional child reduces employment
  probability by roughly zz percentage points around this range (see also Figure~Y).

  \def\sym#1{\ifmmode^{#1}\else\(^{#1}\)\fi}
\begin{tabular}{l*{1}{cccc}}
\toprule
                    & Pred. prob.&   Std. err.&95 conf. low&95 conf. high\\
\midrule
0 children          &      0.6355&      0.0393&      0.5584&      0.7125\\
1 child             &      0.6086&      0.0245&      0.5605&      0.6566\\
2 children          &      0.5812&      0.0090&      0.5635&      0.5989\\
3 children          &      0.5534&      0.0071&      0.5395&      0.5672\\
4 children          &      0.5253&      0.0233&      0.4796&      0.5710\\
5 children          &      0.4971&      0.0396&      0.4194&      0.5747\\
\midrule
Observations        &    4.00e+05&            &            &            \\
\bottomrule
\end{tabular}




  \begin{figure}[htbp]
  \centering
  \includegraphics[width=0.8\textwidth]{figures/ivprobit_pr_kidcount.png}
  \caption{Predicted probability of maternal employment by number of children}
  \label{fig:ivprobit-pr-kidcount}
\end{figure}


\end{enumerate}

\newpage

\section*{References}
\begin{itemize}
  \item Angrist, J.D. and Evans, W.N. (1998). “Children and Their Parents’ Labor Supply: Evidence from Exogenous Variation in Family Size.” \emph{American Economic Review}, 88(3): 450–477.
  \item Couch, K.A. and Placzek, D.W. (2010). “Earnings Losses of Displaced Workers Revisited.” \emph{American Economic Review}, 100(1): 572–589.
  \item Duflo, E., Hanna, R., and Ryan, S.P. (2012). “Incentives Work: Getting Teachers to Come to School.” \emph{American Economic Review}, 102(4): 1241–1278.
  \item Jacobson, L.S., LaLonde, R.J., and Sullivan, D.G. (1993). “Earnings Losses of Displaced Workers.” \emph{American Economic Review}, 83(4): 685–709.
\end{itemize}

\end{document}
