\documentclass[12pt,a4paper]{article}

% packages.tex  (limpio, conserva tus paquetes originales)

% Custom vars
\def\gr{...}      % Set the group number cons. across doc
\def\nass{1}    % Set the assignment number cons. across doc
\def\cl{Econometrics }   % Define the class

% --------------------------
% Encoding & page geometry
% --------------------------
\usepackage[utf8]{inputenc}
\usepackage[T1]{fontenc}
\usepackage[
  a4paper,
  total={170mm,257mm},
  left=25mm,
  right=25mm,
  top=25mm,
  bottom=25mm
]{geometry}

% --------------------------
% Graphics, images, svg
% --------------------------
\usepackage{graphicx}
\usepackage{svg}                 
\graphicspath{{Latex/imgs/}}

% --------------------------
% Math, plots, tikz
% --------------------------
\usepackage{amsmath,amssymb}
\usepackage{mathtools}
\usepackage{pgfplots}
\pgfplotsset{width=12cm,compat=newest}
\usepackage{tikz}
\usetikzlibrary{shapes.geometric, arrows, positioning}

% --------------------------
% Figures, captions, tables
% --------------------------
\usepackage[font=small,labelfont=bf]{caption}
\usepackage{subcaption}          
\usepackage{booktabs}
\usepackage{makecell}
\usepackage{multirow}
\usepackage{threeparttable}
\usepackage{tabularx}
\usepackage{float}
\usepackage{wrapfig}
\usepackage{transparent}

% --------------------------
% Code highlighting (conservo ambos: minted y listings)
% --------------------------
% minted requiere --shell-escape al compilar (ver nota abajo).
\usepackage[newfloat]{minted}     % Code highlighting (más potente)
\newenvironment{code}{\captionsetup{type=listing}}{}
\SetupFloatingEnvironment{listing}{name=Code}

\usepackage{listings}            
\lstset{
  language=Python,
  basicstyle=\ttfamily\footnotesize,
  backgroundcolor=\color{gray!10},
  frame=single,
  keywordstyle=\color{blue},
  commentstyle=\color{green!50!black},
  stringstyle=\color{red!70!black}
}

\setminted{
    fontsize=\small,
    breaklines=true,
    linenos=true,
    autogobble=true,
    mathescape=true,
    breakanywhere=true,
    samepage=false
}

% --------------------------
% Utilities
% --------------------------
\usepackage{xcolor}
\usepackage{url}
\usepackage{enumitem}
\usepackage{xspace}
\usepackage{multicol}

% --------------------------
% Header / footer
% --------------------------
\usepackage{fancyhdr}
\fancyhf{}                           
\fancyhead[L]{DSDM - BSE}
\fancyhead[C]{\cl}
\fancyhead[R]{Assignment \nass}
\renewcommand{\headrulewidth}{0.4pt}
\fancyfoot[C]{\thepage}
\pagestyle{fancy}

% --------------------------
% Extras / includes / pdfpages
% --------------------------
\usepackage{pdfpages}
\usepackage{newclude}               % si usas \includeonly/\newinclude
\usepackage{csquotes}

% --------------------------
% Hyperref (una sola carga) + metadata
% --------------------------
\usepackage{hyperref}
\hypersetup{
    pdftitle    = {\cl - Assignment~\nass},
    pdfsubject  = {This is a submission in the DSDM Masters at BSE.},
    pdfauthor   = {Group~\gr},
    pdfcreator  = {Overleaf},
    pdfstartview= FitH
}

% --------------------------
% Bibliografía (biblatex)
% --------------------------
\usepackage[backend=biber, style=authoryear, hyperref=true]{biblatex}
\usepackage{csquotes}               
\DeclareCiteCommand{\cite}
  {\usebibmacro{prenote}}
  {\usebibmacro{citeindex}%
   \printnames{labelname}%
   \space(\printfield{year})}
  {\multicitedelim}
  {\usebibmacro{postnote}}
\DeclareNameAlias{labelname}{family-given}
\renewcommand*{\nameyeardelim}{\addcomma\space}
\AtEveryCitekey{\ifciteseen{}{\defcounter{maxnames}{1}}}
\addbibresource{Latex/chapters/references.bib}

% --------------------------
% Espaciado / estilo de párrafo
% --------------------------
\setlength{\parskip}{0.3\baselineskip}
\setlength{\parindent}{0pt}
\linespread{1.15}

% --------------------------
% List of code (adecuado para article -> uso section)
% --------------------------
\renewcommand{\listoflistings}{
  \cleardoublepage
  \addcontentsline{toc}{section}{List of Code} 
  \listof{listing}{List of Code}
}


\title{Assignment 6}
\author{
  Samuel Fraley \\
  Eric Gutierrez \\
  Corneel Moons
}
\date{November 17, 2025}

\begin{document}

\maketitle
\tableofcontents
\newpage

\section*{Question 1: Randomized Control Trials}

This question is based on Duflo, Hanna, and Ryan (2012), who evaluate whether teacher
monitoring combined with financial incentives can reduce teacher absenteeism and improve
learning in primary schools.

The NGO Seva Mandir operates non-formal primary schools in rural villages of Rajasthan (India).
Before the program, teacher absenteeism was high (around 35\%). In 2003, Seva Mandir
introduced a teacher incentive program in 57 randomly selected schools. A camera system
was installed to monitor teacher attendance, and teachers were paid according to a nonlinear
function of valid teaching days (at least 5 hours of teaching with at least 8 students).

The program generated an immediate and persistent improvement in attendance in treated schools.

\subsection*{Data}

The dataset \texttt{ps1\_q1.csv} is a simplified version of the original data collected for this RCT.
Each observation corresponds to a visit to one of the study schools (identified by \texttt{schid}).
The variable \texttt{time} equals 1 in the month before the program starts (baseline) and is
greater than 1 in months after the program begins. Schools are randomly assigned to a treatment
group (\texttt{treat}=1) or control group (\texttt{treat}=0). The main outcome variables are
the number of students (\texttt{students}) and teacher attendance (\texttt{teacher\_attendance}).

\subsection*{1.1 Baseline and Experiment Integrity}

Under proper randomization, potential outcomes are independent of treatment status:
\[
Y_{1i}, Y_{0i} \perp\!\!\!\perp D_i,
\]
which implies that the average treatment effect on the treated (ATT) and the average treatment
effect (ATE) coincide: \(\alpha_{ATT} = \alpha_{ATE} = \beta\).

Before analyzing post-treatment outcomes, we should verify that treated and control schools are
similar at baseline.

\begin{enumerate}[label=(\alph*)]
  \item Using only the observations from the month before the program starts (\texttt{time} = 1),
  compute the average teacher attendance and the average number of students per classroom
  separately for the treatment and control groups.

  Replicate the information in Panels A and B of Table 1 in Duflo et al. (2012): produce baseline
  means by treatment status, report appropriate standard errors, and comment on whether the
  randomization appears to have produced comparable groups.

  \item Briefly discuss whether any observed baseline differences are economically and/or statistically
  significant. Explain why this check is important for interpreting the causal effect estimates later.
\end{enumerate}

\subsection*{1.2 Results}

When randomization is valid, the treatment effect can be estimated via the simple regression:
\[
Y_i = \alpha + \beta D_i,
\]
where \(Y_i\) is the outcome of interest (e.g.\ teacher attendance) and \(D_i\) is the treatment
indicator.

\begin{enumerate}[label=(\alph*), resume]
  \item Using post-program data (\texttt{time} > 1), compute and compare average teacher attendance
  in treated and control schools. Replicate the first three columns of Panel A of Table 2 in Duflo
  et al. (2012), reporting the relevant means and differences.

  \item Based on your estimates, discuss whether the incentive program achieved its main goal of
  reducing teacher absenteeism. Comment on the magnitude and statistical significance of the
  estimated treatment effect.
\end{enumerate}

\newpage

\section*{Question 2: Matching}

Jacobson, LaLonde, and Sullivan (1993, JLS) study earnings losses following job displacement.
Using administrative data from Pennsylvania, they document that workers involved in mass
employment reductions suffer long-term earnings losses of roughly 25\% per year. They distinguish
between separations due to mass layoffs and other separations, and use stayers as a control group.

Their model can be written as:
\[
w^A_{it} = \mu_i
+ \sum_{k \ge -4}^{6} \phi_k L_{it}^k
+ \sum_{l \ge -4}^{6} \psi_l M_{it}^l
+ \beta' X_{it}
+ \rho_t + \varepsilon_{it},
\]
where \(w^A_{it}\) is log annual earnings of worker \(i\), \(L_{it}^k\) and \(M_{it}^l\) are sets of
dummies indicating years relative to layoff and mass layoff, \(X_{it}\) is a vector of covariates,
and \(\rho_t\) are time effects.

Couch and Placzek (2010, CP) revisit this question using matching estimators, arguing that
displaced workers are systematically selected, so estimates based only on JLS-type comparisons
may be biased upward.

Let \(D_i=1\) if worker \(i\) is displaced (due to a mass layoff or other separation) and
\(D_i=0\) otherwise, and let \(p(X_i)\) denote the propensity score. The average treatment
effect on the treated (ATT) is:
\[
\alpha_{TT} = \mathbb{E}\Big[
\mathbb{E}[w^A_{1i} \mid D_i = 1, p(X_i)]
- \mathbb{E}[w^A_{0i} \mid D_i = 0, p(X_i)]
\;\Big|\; D_i = 1
\Big].
\]

To compare outcomes relative to a reference year \(t_0\), CP consider a differenced ATT:
\[
\alpha^{D}_{ATT} = \mathbb{E}\Big[
\big(\mathbb{E}[w^A_{1it} \mid D_i = 1, p(X_i)]
      - \mathbb{E}[w^A_{1it_0} \mid D_i = 1, p(X_i)]\big)
-
\big(\mathbb{E}[w^A_{0it} \mid D_i = 0, p(X_i)]
      - \mathbb{E}[w^A_{0it_0} \mid D_i = 0, p(X_i)]\big)
\;\Big|\; D_i = 1
\Big].
\]

Your task is to revisit CP’s findings using a different dataset.

\subsection*{Data}

The dataset \texttt{ps1\_q2.dta} is built from the Veneto Workers Histories (VWH), an administrative
panel including all individuals working in the Italian region of Veneto from 1975–2001. The file
\texttt{ps1\_q2.dta} contains a subsample of workers who, in 1999, either:
\begin{itemize}
  \item experienced a mass employment reduction,
  \item separated from the firm without being part of a mass layoff, or
  \item stayed with the same employer.
\end{itemize}
The panel covers the years 1995–2001. Mass layoffs are defined using the endogenous separation
rate, following JLS and von Wachter, Song, and Manchester (2009). Displaced workers satisfy the
standard requirements in this literature.

\subsection*{2.1 Propensity Score Index}

\begin{enumerate}[label=(\alph*)]
  \item For a year of your choice, estimate the propensity score \(p(X_i)\) using gender, decile of
  1995 earnings, and decade of birth as covariates \(X_i\). Use the \texttt{pscore} command (or an
  equivalent implementation) to compute the propensity scores.

  \item Check the balancing property of the propensity score: verify that, within propensity-score
  strata, the distribution of covariates is similar between displaced and non-displaced workers.
  Summarize and comment on the output.
\end{enumerate}

\subsection*{2.2 Nearest Neighbor and Kernel Matching}

\begin{enumerate}[label=(\alph*), resume]
  \item Using the estimated propensity scores, compute \(\alpha_{ATT}\) with nearest neighbor (NN)
  matching for each year before and after displacement. Clearly indicate the reference year.

  \item Repeat the estimation using kernel matching. In this case, obtain standard errors via
  bootstrap (with at least 200 replications). Report the ATT estimates and their standard
  errors for each year.

  \item Compare the NN and kernel results. Comment on differences in the estimated effects
  and in the associated uncertainty.
\end{enumerate}

\subsection*{2.3 Presenting Results}

\begin{enumerate}[label=(\alph*), resume]
  \item Plot the time path of your estimated effects (before and after displacement) for both NN and
  kernel ATT estimates. Compare your figures to Figure 1 (included with the VWH data description)
  and discuss similarities and differences in the pattern of earnings losses.
\end{enumerate}

\newpage

\section*{Question 3: Instrumental Variables}

Angrist and Evans (1998) use an IV strategy to analyze how the number of children affects
parents’ labor supply. They find a sizable negative effect for mothers and essentially no effect
for fathers. Here we focus on mothers and on employment (rather than hours worked).

\subsection*{Data}

The dataset \texttt{ps1\_q3.dta} is a subset of the data used by Angrist and Evans (1998) and
contains only mothers. The key variables are:
\begin{itemize}
  \item \texttt{sexk}: sex of the first child,
  \item \texttt{kidcount}: total number of children,
  \item \texttt{agem}: age of the mother,
  \item \texttt{twin\_latest}: indicator equal to 1 if the last birth was a twin birth,
  \item \texttt{blackm}, \texttt{hispm}, \texttt{othracem}: race dummies,
  \item \texttt{workedm}: indicator equal to 1 if the mother is employed.
\end{itemize}

\subsection*{3.1 Baseline Models}

Consider the model:
\[
y_i = \beta_0 + \beta_1 \text{kidcount}_i + X_i' \beta + \varepsilon_i,
\]
where \(y_i\) is the mother’s employment status and \(X_i\) is a vector of controls.

\begin{enumerate}[label=(\alph*)]
  \item Estimate this equation using OLS.

  \item Estimate the same specification using a probit model.

  \item Discuss whether these approaches (OLS and probit) are appropriate for identifying the causal
  effect of the number of children on mothers’ labor supply. Be specific about possible sources
  of endogeneity and functional-form issues.
\end{enumerate}

\subsection*{3.2 IV Probit}

\begin{enumerate}[label=(\alph*), resume]
  \item Re-estimate the model using an IV probit specification, instrumenting \texttt{kidcount} with
  \texttt{twin\_latest}. Clearly state the first-stage and structural equations.

  \item Explain the economic intuition behind using twin births as an instrument. Discuss the
  relevance and validity (exclusion restriction) of this instrument, and provide a critical assessment
  of potential threats to its validity.
\end{enumerate}

\subsection*{3.3 Marginal Effects}

\begin{enumerate}[label=(\alph*), resume]
  \item Using your preferred IV probit specification, estimate the marginal effect of an additional
  child on the probability that a mother is employed.

  \item Plot how this marginal effect varies with relevant covariates (e.g.\ age or baseline number
  of children), or report marginal effects evaluated at meaningful covariate profiles. Comment
  on the pattern of these effects and what they imply about the impact of family size on
  mothers’ employment.
\end{enumerate}

\newpage

\section*{References}
\begin{itemize}
  \item Angrist, J.D. and Evans, W.N. (1998). “Children and Their Parents’ Labor Supply: Evidence from Exogenous Variation in Family Size.” \emph{American Economic Review}, 88(3): 450–477.
  \item Couch, K.A. and Placzek, D.W. (2010). “Earnings Losses of Displaced Workers Revisited.” \emph{American Economic Review}, 100(1): 572–589.
  \item Duflo, E., Hanna, R., and Ryan, S.P. (2012). “Incentives Work: Getting Teachers to Come to School.” \emph{American Economic Review}, 102(4): 1241–1278.
  \item Jacobson, L.S., LaLonde, R.J., and Sullivan, D.G. (1993). “Earnings Losses of Displaced Workers.” \emph{American Economic Review}, 83(4): 685–709.
\end{itemize}

\end{document}
